\section{Цели}
\begin{enumerate}
\item[$\bullet$] Сравнить результаты работы алгоритмов выравнивания ClustalW и Muscle на примере Ribosomal protein L7
\item[$\bullet$] Сравнить топологии деревьев, полученные в результате различных методов построения
\item[$\bullet$] Сравнить бутстрэп значения
\item[$\bullet$] Сравнить деревья с принятыми деревьями видов
\end{enumerate}


\section{Скачивание данных}
\subsection*{О данных}
Данные используемые в данной работе взяты с сайта Национального центра биотехнической информации США \href{https://www.ncbi.nlm.nih.gov/}{NCBI}.\\
Данные об исследуемом белке могут быть найдены через поисковит белков \href{https://www.ncbi.nlm.nih.gov/protein/}{NCBI}\\
Также доступна \href{https://www.ncbi.nlm.nih.gov/guide/proteins/#howtos}{документация}.\\

\subsection*{Скачивание}

Скачаем последовательности в формате FASTA с NCBI:
\begin{enumerate}
\item[(1)] Человек -- \href{https://www.ncbi.nlm.nih.gov/protein/AAA03081.1}{Homo sapiens}
\item[(2)] Обезьяна -- \href{https://www.ncbi.nlm.nih.gov/protein/NP_001180485.1}{Macaca mulatta}
\item[(3)] Крыса -- \href{https://www.ncbi.nlm.nih.gov/protein/NP_001094004.1}{Rattus norvegicus}
\item[(4)] Мышь -- \href{https://www.ncbi.nlm.nih.gov/protein/AAA40069.1}{Mus musculus}
\item[(5)] Копытное -- \href{https://www.ncbi.nlm.nih.gov/protein/AAX46363.1}{Bos taurus}
\item[(6)] Сумчатое -- \href{https://www.ncbi.nlm.nih.gov/protein/XP_020826680.1}{Phascolarctos cinereus}
\item[(7)] Змея -- \href{https://www.ncbi.nlm.nih.gov/protein/XP_015672644.1}{Protobothrops mucrosquamatus}
\item[(8)] Ящерица -- \href{https://www.ncbi.nlm.nih.gov/protein/XP_028592540.1}{Podarcis muralis}
\item[(9)] Черепаха -- \href{https://www.ncbi.nlm.nih.gov/protein/EMP32282.1}{Chelonia mydas}
\item[(10)] Птица -- \href{https://www.ncbi.nlm.nih.gov/protein/PKK33771.1}{Columba livia}
\item[(11)] Рыба -- \href{https://www.ncbi.nlm.nih.gov/protein/AAS66968.1}{Danio rerio}
\item[(12)] Растение -- \href{https://www.ncbi.nlm.nih.gov/protein/AAW50989.1}{Triticum aestivum}
\item[(13)] Пекарские дрожжи -- \href{https://www.ncbi.nlm.nih.gov/protein/AAA34982.1}{Saccharomyces cerevisiae}
\item[(14)] Архея -- \href{https://www.ncbi.nlm.nih.gov/protein/AAL64858.1}{Pyrobaculum aerophilum}
\item[(15)] Бактерия -- \href{https://www.ncbi.nlm.nih.gov/protein/AAA26811.1}{Streptomyces antibioticus}
\end{enumerate}

\newpage
\section{Обработка данных}
\subsection*{Выравнивание}
Выравнием последовательности методом ClustalW и Muscle, получим выровненные последовательности \href{https://drive.google.com/file/d/1HtNPjupakaTaff6-jQJAMWBIQIxa2x7V/view?usp=sharing}{ClustalW} и \href{https://drive.google.com/file/d/1UYL78IIqOBh-07wvJQBQxqB1CJfv0uhB/view?usp=sharing}{Muscle}.
\vskip 0.1in
Заметим, что они различны:

\begin{figure}[!h]
	\includegraphics[width=1\linewidth]{Pics/RL7_ClustalW.png}
	\caption{ClustalW}
\end{figure}

\begin{figure}[!h]
	\includegraphics[width=1\linewidth]{Pics/RL7_Muscle.png}
	\caption{Muscle}
\end{figure}


\subsection*{Деревья}
Построим деревья, используя MEGA X\\
Выставим параметры Test of Phylogeny -- Bootstrap method; No. of Bootstrap Replications -- 100.\\
Затем экспортируем полученные деревья в Newick и отформатируем их в Fig tree.\\
Для каждого из деревьев приведена ссылка на исходный Newick.

\subsubsection*{ClustalW}
\begin{figure}[!h]
	\includegraphics[width=1\linewidth]{Tree_ClustalW/RL7_ClustalW_UPGMA.png}
	\caption{\href{https://drive.google.com/file/d/1hPpDUa9uJmDLcnrepNsUhFL9HPq9cWZD/view?usp=sharing}{UPGMA}}
\end{figure}
\newpage
\begin{figure}[!h]
	\includegraphics[width=1\linewidth]{Tree_ClustalW/RL7_ClustalW_NJ.png}
	\caption{\href{https://drive.google.com/file/d/1fIEN3s1rEiak2htmVUBiG3kB9EShOhEs/view?usp=sharing}{Neighbor Joining}}
\end{figure}
\begin{figure}[!h]
	\includegraphics[width=1\linewidth]{Tree_ClustalW/RL7_ClustalW_ML.png}
	\caption{\href{https://drive.google.com/file/d/1OAH_Js0PMo2zVixTMYlEwZyR-ohE5Rn8/view?usp=sharing}{Maximum Likelihood}}
\end{figure}
\newpage
\subsubsection*{Muscle}

\begin{figure}[!h]
	\includegraphics[width=1\linewidth]{Tree_Muscle/RL7_Muscle_UPGMA.png}
	\caption{\href{https://drive.google.com/file/d/11cxH8ehwfc-_q8jbqnSrhElHRs3Ktvhq/view?usp=sharing}{UPGMA}}
\end{figure}
\begin{figure}[!h]
	\includegraphics[width=1\linewidth]{Tree_Muscle/RL7_Muscle_NJ.png}
	\caption{\href{https://drive.google.com/file/d/1wzBnoBK5oBEDgvbvZISxCxtlgAArmOFH/view?usp=sharing}{Neighbor Joining}}
\end{figure}
\newpage
\begin{figure}[!h]
	\includegraphics[width=1\linewidth]{Tree_Muscle/RL7_Muscle_ML.png}
	\caption{\href{https://drive.google.com/file/d/1Mw3AzA7iSqX0yH_KnsdqV6_m5NO675vo/view?usp=sharing}{Maximum Likelihood}}
\end{figure}
\newpage


\section{Выводы}
\begin{enumerate}
\item[$\bullet$] Muscle, так как он помимо глобального выравнивания, выравнивает локально\\
	Но при построении эти различия выявились только в методе UPGMA. Возможно, при большем количестве видов и большем значении bootstrap различия будут более заметными.
\item[$\bullet$] Нет, топология деревьев различается.\\
	Дерево, полученное в результате сортировки ClustalW и метода построения UPGMA отличается от всех остальных расположением птицы(Columba livia) и черепахи(Chelonya mydas). Все остальные деревья изоморфны
\item[$\bullet$] Нет, значения различны.
\item[$\bullet$] Нет, не совпадают, например во всех деревьях копытное к примату ближе, чем человек.
\end{enumerate}
